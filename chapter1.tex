\documentclass[lambda.tex]{subfiles}
\begin{document}

\paragraph{Введение.} ~\\

\begin{tcolorbox}
VODAVODAVODA
$\rb$
$\rbm$
\end{tcolorbox}

Опишем лямбда исчисление формально
Множество лямбда-термов строится из бесконечного множества переменных $\mathcal{V}$ использованием аппликации и абстракции:
\[\mathcal{V} = \{ v, v', v'', v''', ... \}\]

При этом:
\begin{align*}
x \in \mathcal{V} & \implies x \in \Lambda & \text{(Переменная является лямбда-термом)}\\
M, N \in \mathcal{V} & \implies M N \in \Lambda & \text{(Аппликация является лямбда-термом)}\\
M \in \mathcal{V}, v \in \mathcal{V} & \implies \lambda v.M \in \Lambda & \text{(Абстракция является лямбда-термом)}
\end{align*}

Или же, используя БНФ:
\begin{align*}
var &::= v|var'\\
\Lambda &::= var | (\textit{var var}) | ( \lambda var . \Lambda )
\end{align*}

В дальнейшем условимся, что:
\begin{itemize}
\item\(x,y,z,\dots - \text{произвольные переменные.}\)
\item\(M,N,L,\dots - \text{произвольные лямбда термы.}\)
\item Скобки верхнего уровня опускаются.
\item Аппликация правоассоциативна, т.е.\\
\[F\ M_1 \ M_2 \ \dots\ M_n \equiv (\dots((F\ M_1)\ M_2)\ \dots\ M_n)\]
\item Соответственно для абстракции справедливо
\[\lambda x_1 \ x_2 \ \dots\ x_n .M \equiv \lambda x_1 .(\lambda x_2 .\ \dots\ (\lambda x_n .M))\]
\end{itemize}

Множеством свободных переменных терма $N$ называется $FV(N)$, индуктивно определяемое по следующим правилам:
\begin{align*}
FV(x) &\equiv \{x\}\\
FV(M\ N) &\equiv FV(M)\cup FV(N)\\
FV(\lambda x.N) &\equiv FV(N)\setminus\{x\}
\end{align*}

Мы будем называть переменную связанной, если она не принадлежит множеству свободных. Множество связанных переменных терма $N$ принято обозначать как $BV(N)$ Заметим, что переменная связана, если она образует абстракцию.

Мы будем называть терм $M$ закрытым (или комбинатором), если\\ \(FV(M) \equiv \varnothing\). Множество комбинаторов обозначим как $\Lambda^\circ$.

\begin{tcolorbox}
Введем на $\Lambda$ отношение эквивалентности, задаваемое следующим образом:
\begin{align*}
\forall P &=_\alpha P\\
\lambda x.P &=_\alpha \lambda y.P[x:=y] \text{ if } y \not\in FV(P)
\end{align*}

Это отношение носит название $\alpha$-эквивалентность. Так же напишем некоторые аксиомы для этого отношения:

\begin{align*}
M &=_\alpha M\\
M =_\alpha N &\Rightarrow N =_\alpha M\\
M =_\alpha N, N =_\alpha L &\Rightarrow M =_\alpha L\\
\\
M =_\alpha M' &\Rightarrow M\ Z =_\alpha M'\ Z\\
M =_\alpha M' &\Rightarrow Z\ M =_\alpha Z\ M'\\
M =_\alpha M' &\Rightarrow \lambda x.M =_\alpha \lambda x.M'
\end{align*}
Если $M =_\alpha N$, иногда также пишут $\lambda\models M =_\alpha N$

%TODO: перепилить как правила вывода

Обобщая определения выше, $M$ и $N$ равны (альфа-эквивалентны), если можно получить один из другого, путем замены имен связанных переменных. Любые два равных терма в одинаковом контексте так же будут равны.
%Это так ты про альфа-эквивалентность коротко? Вот пидор.
%[Д] Тип да. Так лучше?

Сам процесс замены имени носит название $\alpha$-конверсия и определяется следующим образом:
\begin{equation*}
\lambda x.M \rightarrow_\alpha \lambda y.(M[x := y]) if y \not\in FV(M)\tag{$\alpha$}
\end{equation*}
% запили как с бетой
% DONE



\end{tcolorbox}

Результат подстановки $N$ вместо всех свободных вхождений $x$ в $M$ обозначим $M[x := N]$ и определим как:
\begin{align*}
x[x := N] &\equiv N\\
y[x := N] &\equiv y\ (\text{если } x \neq y)\\
(M_1 \ M_2 )[x := N] &\equiv ((M_1 [x := N])\ (M_2 [x := N]))\\
(\lambda y.M)[x := N] &\equiv (\lambda y.M[x := N])
\end{align*}

%TODO: Что-то про Variable convention
\begin{tcolorbox}

Условимся, что подстановка всегда выполняется корректно, то есть, заменяемая переменная никогда не является связанной ни в каких внутренних термах. Этой ситуации всегда можно избежать заменой имен во внутреннем терме. Например для следующей подстановки предварительно произведем замену имени во внутренней лямбды
\[\lambda a.\lambda b.a\ b[a := b] = \lambda a.(\lambda b.a\ b[b := b'])[a := b] = \lambda b.\lambda b'.b\ b'\]


\end{tcolorbox}

Теперь мы готовы описать $\lambda$-исчисление как формальную теорию.\\

Основная схема $\lambda$-исчисления
\begin{equation*}
\forall M, N \in \Lambda : (\lambda x.M)\ N = M[x := N]\tag{$\beta$}
\end{equation*}

\newtheorem{lemma}{Лемма}
\begin{lemma}
\[(\lambda x_1 \ x_2 \ \dots\ x_n .M)\ X_1 \ X_2 \ \dots\ X_n \equiv M[x_1 := X_1 ][x_2 := X_2]\dots[x_n := X_n ]\]
\end{lemma}
\begin{proof}
Пусть $M' = \lambda x_2 \dots x_n .M$. По аксиоме $(\beta)$ мы имеем
\[(\lambda x_1 .M')\ X_1 \ X_2 \ \dots\ X_n \equiv M'[x_1 := X_1 ] X_2 \ \dots\ X_n\]
Дальнейшее равенство получаем индукцией по связанным переменным.
\end{proof}

\begin{tcolorbox}
Однократное применение аксиомы ($\beta$) будем называть $\beta$-редукцией и обозанчать как ($\rb$)
Применение аксиомы ($\beta$) ноль или более раз обозначим как $\rbm$
Также введем отношение $\beta$-эквивалентности стандартным образом, как транзитивное замыкание отношения $\rbm$ и обозначим $=_\beta$
%TODO: корректность этой фигни
\end{tcolorbox}

\begin{tcolorbox}
На данном этапе мы можем относительно свободно строить различные термы, однако особый подход требуется для описания рекурсивный функций, о чем сейчас и пойдет речь. Для бестипового лямбда-исчисления справедлива следующая теорема:
% [Теорема о неподвижной точке]
\newtheorem*{fixpoint}{Теорема о неподвижной точке}
\begin{fixpoint}
\begin{align*}
&(i)\ \ \forall F. \exists X. (F\ X = X) \\
&\text{Более того, существует комбинатор, находящий $X$}\\
% &\text{}\\
&\ \ \ \ \ Y = \lambda f.(\lambda x.f\ (x\ x)) (\lambda x.f\ (x\ x))\\
&(ii)\ \forall F.(Y\ F=F\ (Y\ F))
\end{align*}
\end{fixpoint}

\begin{proof}
~\\
Определим $W = \lambda x.F(x\ x)$ и $X = W\ W$ тогда
\begin{align*}
X \equiv W\ W \equiv (\lambda x.F(x\ x))W \rightarrow_\beta\\
F\ (W\ W) \equiv F X
\end{align*}
Аналогично
\begin{align*}
&Y\ F \rightarrow_\beta\\
&(\lambda x.F\ (x\ x)) (\lambda f.(\lambda x.F\ (x\ x))) \rightarrow_\beta\\
&F\ ((\lambda f.(\lambda x.F\ (x\ x))) (\lambda f.(\lambda x.F\ (x\ x)))) \equiv F\ (Y\ F)
\end{align*}
\end{proof}


\end{tcolorbox}

\end{document}