\documentclass{article}
\usepackage[T2A]{fontenc}
\usepackage[utf8x]{inputenc}
\usepackage[english, russian]{babel} % Localisation

% \usepackage{fontspec} % loaded by polyglossia, but included here for transparency 
% \usepackage{polyglossia}
% \setmainlanguage{russian} 
% \setotherlanguage{english}

\usepackage{amssymb} % More math symbols
\usepackage{amsmath} % Math constructions
\usepackage{amsthm} % Theorems
\usepackage{tcolorbox} % Dima boxes shit
\usepackage{subfiles} % File separation

% \usepackage[cache=false]{minted}
% \usepackage{minted} % Oh yes.

\newcommand{\rb}{\rightarrow_\beta}
\newcommand{\rbm}{\twoheadrightarrow_\beta}

\begin{document}

\title{Краткое введение в лямбда-исчисление}
\author{Белкин Дмитрий, студент группы 4362
\and Бертыш Вадим, студент группы 4374}
\date{23 декабря 2015}
\maketitle
\newpage

\subfile{chapter1}
\subfile{chapter2}

% TODO: !!!!!!!!!!!!!!!!!!!!!!!!!!!!!!!!!!
% [Теорема Чёрча-Россера]

% [Вставить про комбинаторы]
% [Нормализуемость]
% [Стратегии вычисления]
% [Полнота по тьюрингу]



\end{document}
