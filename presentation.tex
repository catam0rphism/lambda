\documentclass{beamer}
\usepackage[T2A]{fontenc}
\usepackage[utf8x]{inputenc}
\usepackage[english, russian]{babel} % Localisation

% \usepackage{fontspec} % loaded by polyglossia, but included here for transparency 
% \usepackage{polyglossia}
% \setmainlanguage{russian} 
% \setotherlanguage{english}

\usepackage{amssymb} % More math symbols
\usepackage{amsmath} % Math constructions
\usepackage{amsthm} % Theorems
\usepackage{tcolorbox} % Dima boxes shit
\usepackage{subfiles} % File separation
\usepackage[all]{xy} % Diagrams

% \usepackage{minted} % Oh yes.

% Beta reduction
\newcommand{\rb}{\rightarrow_\beta}
\newcommand{\rbm}{\twoheadrightarrow_\beta}
\newcommand{\beteq}{=_\beta}

% Eta reduction
\newcommand{\rn}{\rightarrow_\eta}
\newcommand{\rnm}{\twoheadrightarrow_\eta}
\newcommand{\eteq}{=_\eta}

% Terms for math mode
\newcommand{\mmterm}[1]{\text{\textbf{#1}}}
\newcommand{\mmnot}{\mmterm{not}}
\newcommand{\mmand}{\mmterm{and}}
\newcommand{\mmor}{\mmterm{or}}
\newcommand{\mmif}{\mmterm{if}}
\newcommand{\mmt}{\mmterm{True}}
\newcommand{\mmf}{\mmterm{False}}
\newcommand{\mmchurchn}[1]{\boldsymbol{\overline{#1}}}

\usetheme{Rochester}
\usecolortheme{rose}

\title[Лямбда исчисление]{Краткое введение в лямбда исчисление}
\author[Белкин, Бертыш]{Д.~Белкин \and В.~Бертыш}

\begin{document}

\frame{\titlepage}

\begin{frame}\frametitle{Формальное определение}

Множество лямбда-термов строится из бесконечного множества переменных $\mathcal{V}$ использованием аппликации и абстракции:
\[\mathcal{V} = \{ v, v', v'', v''' \dots \}\]

\begin{align*}
	x \in \mathcal{V} & \implies x \in \Lambda &\text{(Переменная является $\lambda$-термом)}\\
	M, N \in \mathcal{V} & \implies M\ N \in \Lambda &\text{(Аппликация является $\lambda$-термом)}\\
	M \in \mathcal{V}, v \in \mathcal{V} & \implies \lambda v.M \in \Lambda &\text{(Абстракция является $\lambda$-термом)}
\end{align*}

\end{frame}

\begin{frame}\frametitle{Свободные и связанные переменные}

Множество свободных переменных терма $N$ обозначается $FV(N)$

\begin{align*}
	FV(x) 			&\equiv \{x\}\\
	FV(M\ N) 		&\equiv FV(M)\cup FV(N)\\
	FV(\lambda x.N) &\equiv FV(N)\setminus\{x\}
\end{align*}
    
Множество связанных переменных терма $N$ принято обозначать как $BV(N)$.
Заметим, что переменная связана, если она образует абстракцию.
Мы будем называть терм $M$ закрытым (или комбинатором), если \(FV(M) \equiv \varnothing\).
Множество комбинаторов обозначим как $\Lambda^\circ$.

\end{frame}

\begin{frame}\frametitle{Подстановки}
    
Результат подстановки $N$ вместо всех свободных вхождений $x$ в $M$ обозначим $M[x := N]$ и определим как:
\begin{align*}
	x[x := N] &\equiv N\\
	y[x := N] &\equiv y\ (\text{если } x \neq y)\\
	(M_1 \ M_2 )[x := N] &\equiv ((M_1 [x := N])\ (M_2 [x := N]))\\
	(\lambda y.M)[x := N] &\equiv (\lambda y.M[x := N])
\end{align*}

\end{frame}

\begin{frame}\frametitle{$\alpha$-конверсия}
    
Введем на $\Lambda$ отношение эквивалентности, задаваемое следующим образом:
\begin{align*}
	\forall P\ P &=_\alpha P\\
	\lambda x.P &=_\alpha \lambda y.P[x:=y] \text{, если } y \not\in FV(P)
\end{align*}

Это отношение назвается $\alpha$-эквивалентностью.

Если $M =_\alpha N$, иногда также пишут $\lambda\models M =_\alpha N$

\end{frame}

\begin{frame}\frametitle{$\beta$-редукция}
    
Основная схема $\lambda$-исчисления
\begin{equation*}
	\forall M, N \in \Lambda : (\lambda x.M)\ N = M[x := N]\tag{$\beta$}
\end{equation*}
Однократное применение аксиомы ($\beta$) будем обозначать как ($\rb$).
Место, где можно применить аксиому ($\beta$) будем называть редексом.
Терм, в котором нет редексов будем называть нормальной формой выражения.
Применение аксиомы ($\beta$) ноль или более раз (транзитивное замыкание $\rb$) будем называть $\beta$-редукцией и обозначим как $\rbm$.

Также введем отношение $\beta$-эквивалентности, которое обозначим как $=_\beta$ и определим как
\begin{align*}
	&\forall M,N \in \Lambda: M \rbm N \implies M=_\beta N
\end{align*}

\end{frame}

\begin{frame}\frametitle{$\eta$-редукция}
    
Дополним наше $\lambda$-исчисление еще одной аксиомой
\begin{equation*}
	\lambda x.M\ x \equiv M \text{ если }x \not\in FV(M) \tag{$\eta$}
\end{equation*}
Аналогично $\beta$-редукции вводятся однократное применение аксиомы $\eta$ ($\rn$), её транзитивное замыкание -- $\eta$-редукция ($\rnm$) и отношение $\eta$-эквивалентности ($=_\eta$).

\end{frame}

\begin{frame}\frametitle{Теорема о неподвижной точке}
    
\begin{align*}
	&(i)\ \ \forall F\ \exists X\ (F\ X = X) \\
	&\text{Более того, существует комбинатор, находящий $X$}\\
	% &\text{}\\
	&\ \ \ \ \ \boldsymbol{Y} = \lambda f.(\lambda x.f\ (x\ x)) (\lambda x.f\ (x\ x))\\
	&(ii)\ \forall F\ \boldsymbol{Y}\ F=F\ (\boldsymbol{Y}\ F)
\end{align*}

\end{frame}

\begin{frame}\frametitle{Теорема Чёрча-Россера}
    
	$$
	\forall M,N_1,N_2:M
	\rbm N_1, M
	\rbm N_2
	\implies
	\exists N : N_1
	\rbm N,N_2
	\rbm N
	$$
	Иначе говоря, для $\beta$-редукции выполняется свойство ромба
	\begin{align*}
		\xymatrix{
		& \ar@{->>}[dl]_\beta M \ar@{->>}[dr]^\beta \\
		N_2 \ar@{->>}[dr]^\beta & & N_1 \ar@{->>}[dl]_\beta \\
		&N
	}
	\end{align*}

\end{frame}

\begin{frame}\frametitle{Логические значения}
    
Пусть \textbf{True} и \textbf{False} некие лямбда термы, при этом $\mmt \neq_\beta \mmf$. Один из возможных способов сделать это следующий

\begin{align*}
\mmt = \lambda a b.a\\
\mmf = \lambda a b.b
\end{align*}

\end{frame}

\begin{frame}\frametitle{Логические операции}
    
Определим термы \textbf{and}, \textbf{or} и \textbf{not}, представляющие соответствующие логические операции

\begin{align*}
&\mmand = \lambda t_1 t_2 a b . (t_1\ (t_2\ a\ b)\ b)\\
&\mmor = \lambda t_1 t_2 a b.(t_1\ a\ (t_2\ a\ b))\\
&\mmnot = \lambda f a b . f\ b\ a
\end{align*}

\end{frame}

\begin{frame}\frametitle{Ветвление}
    
Условную конструкцию if можно определить следующим образом
\begin{align*}
	&\mmif = \lambda t. t\\
	&\mmif\ \mmt\ a\ b \rbm a\\
	&\mmif\ \mmf\ a\ b \rbm b\\
\end{align*}

\end{frame}

\begin{frame}\frametitle{Натуральные числа}
    
Определим натуральные числа следующим образом:
\begin{align*}
	&\mmchurchn{0} = \lambda fx.x = \mmterm{const}\\
	&\mmchurchn{1} = \lambda fx.f\ x = \mmterm{id}\\
	&\dots\\
	&\mmchurchn{n} = \lambda f.\lambda x.f^n x\\
	&\text{Где }f^n(x) = \underbrace{f(f(f(\dots f(}_{n\text{ раз}}x))))
\end{align*}

Такое представление чисел называется нумералами Чёрча.

\end{frame}

\begin{frame}\frametitle{Простые арифметические операции}
    
Определим инкремент следующим образом:
\begin{align*}
	& \mmterm{succ}\ \mmchurchn{n} \equiv \mmchurchn{n+1}\\
	& \mmterm{succ} = \lambda nfx. f\ (n\ f\ x)
\end{align*}
Сложение, умножение и возведение в степень определяются так:
\begin{align*}
	&\mmterm{add} = \lambda abfx.(b\ f\ (a\ f)\ x)\\
	&\mmterm{mul} = \lambda abf.a\ (b\ f)\\
	&\mmterm{pow} = \lambda ab.b\ a
\end{align*}

\end{frame}

\begin{frame}\frametitle{Сложные арифметические операции}
    
Вычитание -- сложно.
Нужно для индуктивно построенного $\mmchurchn{n}$ найти $\mmchurchn{n-1}$.

Решение -- пары.
\begin{align*}
	\mmterm{mkPair} = &\lambda abf.f\ a\ b\\
	\mmterm{fst} = &\lambda p.p\ \mmterm{True}\\
	\mmterm{snd} = &\lambda p.p\ \mmterm{False}
\end{align*}
Здесь $\mmterm{mkPair}$ создает пару, а $\mmterm{fst}$ и $\mmterm{snd}$ получают первый и второй элементы пары соответственно.

Далее определим ``инкремент'' пары
\[\mmterm{succP} = \lambda p. \mmterm{mkPair}\ (\mmterm{snd}\ p)\ (\mmterm{succ}\ (\mmterm{fst}\ p))\]

\end{frame}

\begin{frame}\frametitle{Вычитание}
    
Теперь можно определить декремент
\[\mmterm{prev} = \lambda n.\mmterm{fst}\ (n\ \mmterm{succP}\ (\mmterm{mkPair}\ \mmchurchn{0}\ \mmchurchn{0}))\]

и вычитание
\[\mmterm{sub} = \lambda ab.b\ \mmterm{prev}\ a\]

\end{frame}

\begin{frame}\frametitle{Рекурсивные функции}
    
Попробуем написать терм, вычисляющий факториал.
Для начала нужно определять, что $\mmchurchn{n} = \mmchurchn{0}$.
Напишем терм, выполняющий эту проверку:
\[\mmterm{isZero} = \lambda n.n\ (\lambda x. \mmf)\ \mmt\]

\end{frame}

\end{document}
