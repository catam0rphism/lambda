\documentclass[lambda.tex]{subfiles}

\newcommand{\mmterm}[1]{\text{\textbf{#1}}}
\newcommand{\mmnot}{\mmterm{not}}
\newcommand{\mmand}{\mmterm{and}}
\newcommand{\mmor}{\mmterm{or}}
\newcommand{\mmif}{\mmterm{if}}
\newcommand{\mmt}{\mmterm{True}}
\newcommand{\mmf}{\mmterm{False}}
\newcommand{\mmchurchn}[1]{\boldsymbol{\overline{#1}}}

\begin{document}
\subsection*{Приложение}
\subsubsection*{Задачи}
\paragraph{Свободные и связанные переменные}~\\\\
\textbf{Упражнение 1.}~\\
\textbf{Дано:} Определения множества свободных ($FV$) и связанных ($BV$) переменных, лямбда терм.\\
\textbf{Требуется:} Определить для него множества свободных и связанных переменных.
\settasks{counter-format = tsk)}
\begin{tasks}(1)
	\task $\lambda xy.x\ z\ f\ y$
	\task $\lambda xy.(\lambda y.f\ y)\ x\ f\ s\ y$
	\task $\lambda xs.x\ s\ x$
	\task $a\ f\ (\lambda ab.a\ x\ f\ b)\ m$
	\task $\lambda axs.a\ (\lambda fx.x\ f)\ (s\ x)\ f$
\end{tasks}
\emph{Дополнительно: Есть ли среди них комбинаторы?}\\
\\
\textbf{Упражнение 2.}~\\
\textbf{Дано:} Определения комбинаторов.\\
\textbf{Требуется:} Написать соответствующие лямбда термы.
\begin{tasks}(1)
	\task $\boldsymbol{S}xyz = xz(yz)$
	\task $\boldsymbol{K}xy = x$
    \task $\boldsymbol{I}x = x$
\end{tasks}
\textbf{Задача 1.}~\\
\emph{Используя определение комбинаторов \textbf{S} и \textbf{K} выразите комбинатор \textbf{I}.}
\paragraph{Подстановки}~\\\\
\textbf{Упражнение 3.}\\
\textbf{Дано:} Лямбда терм.\\
\textbf{Требуется:} Провести указанную подстановку.
\begin{tasks}(1)
	\task $(\lambda ab.b\ a)[a:=b]$
	\task $(a\ s)[a:=b\ f\ s]$
	\task $(\lambda a.x\ a\ b)[a:=\lambda g.y\ g]$
	\task $(a\ x\ f)[x:=\lambda f.a\ f\ y]$
\end{tasks}
\paragraph{$\beta$-редукция}~\\\\
\textbf{Упражнение 4.}\\
\textbf{Дано:} Лямбда терм.\\
\textbf{Требуется:} Провести бета-редукцию терма.
\begin{tasks}(1)
	\task $(\lambda x.x\ x)(\lambda x.x\ x)\rb$
	\task $(\lambda fxy.f\ x\ y)(\lambda xy.y\ x)\rb$
	\task $(\lambda f.(\lambda x.f\ (x\ x)) (\lambda x.f\ (x\ x)))\rb$ \emph{(для этого терма два шага)}
	% [Д] А по которому из редексов?
	\task $(\lambda sxf.s\ x)(s\ f)(\lambda st.f\ s\ t)\rb$
\end{tasks}
\textbf{Задача 2.}~\\
\emph{Напишите лямбда терм, который редуцируется в терм ``$n\ i\ l$''.}
\end{document}
